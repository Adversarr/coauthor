% Implementation Details Section
% Key implementation aspects of Seed

\subsection{JSONL-backed Event and Audit Stores}
\label{subsec:jsonl-storage}

Seed persists task lifecycle events and tool audit records in append-only JSONL files under a workspace-local state directory. This enables replay, debugging, and post-hoc analysis without external infrastructure.

\subsection{Agent Runtime and Tool Loop}
\label{subsec:agent-runtime}

The runtime executes task-scoped agents that produce text, reasoning, tool calls, interaction requests, and terminal outcomes. Tool calls are mediated through a registry and executor boundary.

\subsection{UIP Safety Gate}
\label{subsec:uip}

Risky tools are intercepted by a user interaction protocol (UIP) before execution. Users review context and decide whether to approve or reject, ensuring explicit control over high-impact actions.

\subsection{Role-based Agent Team}
\label{subsec:dual-interface}

Seed ships with three specialized agents:
\begin{itemize}
    \item Coordinator Agent: full tool access and delegation.
    \item Research Agent: read-only workspace survey.
    \item Chat Agent: no-tool quick advisory mode.
\end{itemize}
