% Implementation Details Section
% This section describes the key implementation aspects of the CoAuthor system

\subsection{Event Sourcing with JSONL Storage}
\label{subsec:jsonl-storage}

All domain events are persisted in JSONL (JSON Lines) format, providing an append-only log that captures every state transition. This implementation choice offers several advantages:

\begin{itemize}
    \item Human-readable format for debugging and auditing
    \item Efficient append operations without complex database setup
    \item Natural support for event replay and state reconstruction
    \item Easy integration with version control systems
\end{itemize}

\subsection{Agent Runtime and Tool Use Workflow}
\label{subsec:agent-runtime}

The agent runtime implements an AsyncGenerator-based pattern where agents yield outputs incrementally. Each agent operates within a task-scoped runtime that provides:

\begin{enumerate}
    \item \textbf{Tool Integration}: Agents have access to a registry of tools for file operations, command execution, and subtask delegation.
    \item \textbf{Risk-Aware Execution}: Tools are classified as safe (read-only) or risky (write operations). Risky tools trigger User Interaction Points (UIPs) requiring explicit confirmation.
    \item \textbf{Streaming Support}: LLM responses stream in real-time, enabling immediate user feedback and interaction.
\end{enumerate}

\subsection{User Interaction Point (UIP) Design}
\label{subsec:uip}

UIPs serve as controlled checkpoints where user consent is required before executing potentially destructive operations. The design ensures:

\begin{itemize}
    \item Agents remain unaware of risk classifications
    \item The runtime intercepts risky tool calls and presents confirmation dialogs
    \item Users can review operation details before approval
    \item Audit trails capture all UIP interactions for transparency
\end{itemize}

\subsection{Dual Interface Architecture}
\label{subsec:dual-interface}

CoAuthor supports two complementary interfaces:

\textbf{Terminal UI (TUI)}: Built with React and Ink, the TUI provides a keyboard-driven interface optimized for developer workflows. Features include command-based navigation (\texttt{/new}, \texttt{/pause}, \texttt{/continue}), real-time streaming output, and task tree visualization.

\textbf{Web UI}: Built with React 19, TailwindCSS, and Radix UI components, the Web UI offers a graphical interface with enhanced visualization capabilities. It connects to the backend via WebSocket for real-time updates and supports Markdown rendering with syntax highlighting, LaTeX math, and Mermaid diagrams.

Both interfaces communicate with the same underlying infrastructure through a shared HTTP/WebSocket server, ensuring consistent behavior across interaction modes.
