% System Architecture Section
% This section describes the technical design of the CoAuthor system

\subsection{Overview}
\label{subsec:overview}

CoAuthor is built on a hexagonal architecture that separates domain logic from infrastructure concerns. This design enables testability, flexibility, and clear separation of responsibilities.

\subsection{Core Components}
\label{subsec:components}

The system consists of the following key components:

\begin{itemize}
    \item \textbf{Domain Layer}: Contains business logic, event definitions, and type schemas.
    \item \textbf{Application Layer}: Implements use cases through services for tasks, patches, and events.
    \item \textbf{Infrastructure Layer}: Provides adapters for event storage, LLM clients, HTTP/WebSocket servers, and logging.
    \item \textbf{Interface Layer}: Provides dual interfaces for user interaction:
    \begin{itemize}
        \item \textit{Terminal UI (TUI)}: React/Ink-based interface with command system, real-time streaming output, and task tree navigation.
        \item \textit{Web UI}: Modern browser interface built with React, TailwindCSS, and Radix UI components, connected via WebSocket for real-time updates.
    \end{itemize}
\end{itemize}

\subsection{Event Flow}
\label{subsec:event-flow}

All state changes in CoAuthor are captured as domain events and stored in an append-only log. This event-sourced approach provides:

\begin{enumerate}
    \item Complete audit trails of all changes
    \item Ability to reconstruct any historical state
    \item Natural support for temporal queries
    \item Built-in conflict resolution through event ordering
\end{enumerate}
